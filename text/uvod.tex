\chapter{Úvod do problematiky}
Diplomová práca na tému "Multiplatformní aplikace pro správu síťových prvku Mikrotik" sa bude primárne zaoberať samostatným mikrotikom.\\
V prvej časti práci bude definovanie Mikrotiku a jeho funkcií v skratke, popis operačného systému routerOS, porovnanie podobnosti s operačným systémom unix. Ďalej budú popísané aj pokročilé funkcie ako napríklad capsman, internet protocol security (ipsec), kompletný firewall, zoznamy adries v rámci firewallu. V úvodnej časti práci budú prejdené všetky časti mikrotiku a jeho nastavení.\\
V druhej časti práce bude popis aplication programable interface (API) a spôsoboch softvérového riešenia aplikácie pre správu Mikrotikov. Ďalej tu budú popísané všetky možné spôsoby softvérového riešenia a programovania mikrotikov. Táto časť bude tiež obsahovať niečo ohľadom technológie git, popise, čo je git, princíp tzv. commitu a pushu. Rozdiely medzi vetvami, prepínanie medzi vetvami a pridávania zmien.\\
V tretej časti práce sa priblíži popis programovacieho jazyka python, rozdiely medzi verziami python, a jeho základné vlastnosti. Ďalej sa tu popíše knižnica paramiko, ktorá bude použitá na komunikáciu s operačným systémom routerOS. Buú tu popísané vastnosti knižnice, čo obsahuje, hlavné prvky knižnice, spôsob fungovania, použitie základného API a použitie secure socket layer (SSL), použité porty. Taktiež tu bude spomenutý aj úvod do certifikátov a to konkrétne Single Sign-on metódy. \\
V ďalšej časti bude návrh riešenia softvérovej implementácie aplikácie. Bude obsahovať popis, princípy, diagramy, hlavne Unified modeling language (UML), popisy knižníc, jednotlivých tried a modulov. Každý modul bude popísaný sovojou funkcionalitou, parametrami a výstupom s praktickými ukážkami.\\
V ďalšej časti bude použitá implementácia softvérového návrhu riešenia. Bude tu riešenie ako v konzolovej časti, jeho ukážky, test a výsledky. \\
V poslednej časti práce bude ukážka grafického spracovania konzolovej časti aplikácie a ich prepojenia do jednej aplikácie, spoločne s ukážkami kódov, testu  a výsledkov.