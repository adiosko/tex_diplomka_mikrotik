\chapter{Záver}
Diplomová práca sa zaoberala vytvorením užívateľského rozhrania pre prvky mikrotik na rôzne operačné systémy. Primárne funguje na operačný systém Linux a macOS, ale z časti funguje aj na operačný systém Windows, len je potreba doinštalovať konkrétne knižnice a moduly.\\
Práca obsahuje v prvej časti teoretický rozbor API, knižníc a pythonu. Popisuje možnosti inštalácie, princíp fungovania modulov a použité knižnice.\\
V ďalšej časti sa nachádza popis konzolovej časti aplikácie spôsobom výberu jedneého súboru zo zložky v projekte a popisom jeho metód  a reprezentáciou jeho UML diagramu. Pre každú zložku je najskôr globálny popis a následne vybratý jeden súbor zo zložky a ten je detajlne popísaný.\\
V ďalšej časti  sa nachádza grafická časť aplikácie (frontend) realizovaná cez PyQT4 a QT4 designer. Nachádza sa tu popis prostredia, možnosti, výnimky. Ďalej sú tu v skratke popísané funkcie tlačítok a jeho súbory sú uložené na priloženom DVD a na verzovacom systíme github s konkrétnymi repozitármi.\\
V poslednej časti  práce sa nachádzajú návody na inštaláciu na UNIX systémoch  a na Windows systémoch. Ako posledná časť slúži tetsovanie aplikácie. \\
Práca splňuje zadanie a má vytvorené užívateľské rozhranie,  v rámci API je tam možnosť ďalšej práce. Aplikácia predstavuje konfiguračný nástroj na konfiguráciu prvkov mikrotik, ale vzhľad aplikácie je patrične odlišný od originálneho winboxu. Obsahuje systémové nástroje, nastavenie IP a Systémových nástrojov, vzhľad aplikácie je poňatý ale iným spôsobom ako to je vo winboxe. 