\chapter{Záver}
Diplomová práca sa zaoberala vytvorením užívateľského rozhrania pre prvky mikrotik na operačný systém Linux. \\
V prvej časti práce sa nachádza teoretický rozbor API, knižníc a pythonu. Popisuje možnosti inštalácie, princíp fungovania modulov a použité knižnice.\\
V ďalšej časti sa nachádza popis konzolovej časti aplikácie spôsobom výberu jedného súboru zo zložky v projekte a popisom jeho metód  a reprezentáciou jeho UML diagramov. Pre každú zložku je najskôr globálny popis a následne vybratý jeden súbor zo zložky a ten je detailne popísaný.\\
V ďalšej časti  sa nachádza grafická časť aplikácie (frontend) realizovaná cez PyQT4 a QT4 designer. Nachádza sa tu popis prostredia, možnosti, výnimky. Ďalej sú tu v skratke popísané funkcie tlačítok a jeho súbory sú uložené na priloženom DVD a na verzovacom systíme github s konkrétnymi repozitármi.\\
V poslednej časti  práce sa nachádzajú návody na inštaláciu na UNIX systémoch. Ako posledná časť slúži testovanie aplikácie. \\
Aplikácia predstavuje konfiguračný nástroj na konfiguráciu prvkov mikrotik, ale vzhľad aplikácie je patrične odlišný od originálneho winboxu, taktiež aplikácia predsatvuje zjednodušený winbox. Práca používa vlastné rozhranie, ktoré je zjednodušenie súčasného winboxu, neobsahuje jeho všetky prvky. . Dizajn aplikácie je dynamický a prispôsobiteľný dizajnu konkrétneho operačného systému. Vzhľad aplikácie je poňatý ale iným spôsobom ako to je vo winboxe. Obsahuje systémové nástroje, nastavenie IP a systémových nástrojov, smerovania , nastavenia DHCP, možnosti správy zariadenia a ďalších prvkov.   