\section{Uložené git repozitáre}
Obsah súborov na projekt aplikácie nájdeme na repozitári github: \href{//github.com/adiosko/mikrotik/tree/adrian}{mikrotik}\\
Diplomová práca v textovej podobe je uložená na repozitári github: \href{https://github.com/adiosko/tex_diplomka_mikrotik/tree/adrian}{diplomka_tex}
\section{Obsah priloženého DVD}
DVD obsahuje súbory projektu uložené v zložkách:
\begin{itemize}
\item \textbf{bridge} - táto časť obsahuje konfiguráciu bridgu
\item  \textbf{capsman} - táto časť obsahuje konfiguráciu centrálnej obsluhy mikrotik prístupových bodov  ďalšie funkcie
\item \textbf{certs} - obsahuje certifikáty na pripojenie sa na mikrotik pomocou protokolu API-SSL
\item \textbf{Dude} - obsahuje konfiguráciu Dude
\item \textbf{exportToHtml} - časť predstavuje generovanie súboru na analýzu v podobe webovej stránky
\item \textbf{interfaces} - časť predstavuje konfiguráciu rozhraní na mikrotiku
\item \textbf{IPv4} - rozsiahla časť, obsahuje konfiguráciu IP adries, ...
\item \textbf{IPv6} - pre zložku IPv6 platí to isté čo pre zložku IPv4, ale platí pre konfiguráciu na základe IPv6 adresného rozsahu
\item \textbf{KVM} - sekcia obsahuje možnosti virtualizácie mikrotiku
\item \textbf{log} - sekcia obsahuje analýzu a konfiguráciu logu zariadenia
\item \textbf{makeSupportFile} - seckia obsahuje vytvorenie súboru potrebného pre analýzu na mikrotik podpore
\item \textbf{mesh} - sekcia popisuje konfiguráciu tzv. mesh technológie
\item \textbf{MPLS} - sekcia obsahuje možnosti konfiurácie Multi Protocol Label Switching (MPLS)
\item \textbf{PPP} - sekcia obsahuje konfiguráciu Point to Point Protocol (PPP) a ďalších možností Virtual Private Network (VPN) konfigurácie.
\item \textbf{Queues} - sekcia obsahuje konfiguráciu sieťových front
\item \textbf{Radius} - sekcia obsahuje nastavenie autentizácie Radius 
\item \textbf{Routing} - sekcia obsahuje možnosti dynamického smerovania 
\item \textbf{Switch} - sekcia obsahuje konfiguráciu prepínača
\item \textbf{System} - sekcia obsahuje časť konfigurácie systémových nástrojov,
\item \textbf{Tools} - sekcia obshauje konfiguráciu mikrotik nástrojov
\item \textbf{Wireless} - sekcia obsahuje konfiguráciu bezdrátového rozhrania
\item \textbf{loginGui} - sekcia obsahuje súbory pre GUI
\end{itemize}
