% Soubory musí být v kódování, které je nastaveno v příkazu \usepackage[...]{inputenc}
\documentclass[%
%  draft,    				  % Testovací překlad
  12pt,       				% Velikost základního písma je 12 bodů
  a4paper,    				% Formát papíru je A4
%  oneside,      			% Jednostranný tisk (výchozí)
%% Z následujicich voleb lze použít maximálně jednu:
%	dvipdfm  						% výstup bude zpracován programem 'dvipdfm' do PDF
%	dvips	  						% výstup bude zpracován programem 'dvips' do PS
%	pdftex							% překlad bude proveden programem 'pdftex' do PDF (výchozí)
%% Z následujících voleb lze použít jen jednu:
%english,            % originální jazyk je angličtina
%czech              % originální jazyk je čeština (výchozí)
slovak,             % originální jazyk je slovenčina
]{report}				    	% Dokument třídy 'zpráva'

\usepackage[utf8]		%	Kódování zdrojových souborů je Windows-1250
	{inputenc}					% Balíček pro nastavení kódování zdrojových souborů

\usepackage{graphicx} % Balíček 'graphicx' pro vkládání obrázků
											% Nutné pro vložení log školy a fakulty
%\usepackage{thebibliogrpahy}
\usepackage[
	nohyperlinks				% Nebudou tvořeny hypertextové odkazy do seznamu zkratek
]{acronym}						% Balíček 'acronym' pro sazby zkratek a symbolů
											% Nutné pro použití prostředí 'seznamzkratek' balíčku 'thesis'

\usepackage[
	unicode,						% Záložky a informace budou v kódování unicode
	breaklinks=true,		% Hypertextové odkazy mohou obsahovat zalomení řádku
	hypertexnames=false % Názvy hypertextových odkazů budou tvořeny
											% nezávisle na názvech TeXu
]{hyperref}						% Balíček 'hyperref' pro sazbu hypertextových odkazů
											% Nutné pro použití příkazu 'nastavenipdf' balíčku 'thesis'

\usepackage{pdfpages} % Balíček umožňující vkládat stránky z PDF souborů
                      % Nutné při vkládání titulních listů a zadání přímo
                      % ve formátu PDF z informačního systému

\usepackage{enumitem} % Balíček pro nastavení mezerování v odrážkách
  \setlist{topsep=0pt,partopsep=0pt,noitemsep}

\usepackage{cmap} 		% Balíček cmap zajišťuje, že PDF vytvořené `pdflatexem' je
											% plně "prohledávatelné" a "kopírovatelné"

\usepackage{upgreek}	% Balíček pro sazbu stojatých řeckých písmem
											% např. stojaté pí: \uppi
											% např. stojaté mí: \upmu (použitelné třeba v mikrometrech)
											% pozor, grafická nekompatibilita s fonty typu Computer Modern!

%% Nastavení českého jazyka při sazbě v češtině.
% Pro sazbu češtiny je možné použít mezinárodní balíček 'babel', jenž
% použití doporučujeme pro nové instalace (MikTeX2.8,TeXLive2009), nebo
% národní balíček 'czech', který doporučujeme ve starších instalacích.
% Balíček 'babel' bude správně fungovat pouze ve spojení s programy
% 'latex', 'pdflatex', zatímco balíček 'czech' bude fungovat ve spojení
% s programy 'cslatex', 'pdfcslatex'.
% Varianta A:
\usepackage    				
  {babel}             % Balíček pro sazbu různojazyčných dokumentů; kompilovat (pdf)latexem!
  										% převezme si z parametrů třídy správný jazyk
\usepackage{lmodern}	% vektorové fonty Latin Modern, nástupce půvoních Knuthových Computern Modern fontů
\usepackage{float}
\usepackage{textcomp} % Dodatečné symboly
\usepackage[T1]{fontenc}  % Kódování fontu - mj. kvůli správným vzorům pro dělení slov
% Varianta B:
%\usepackage{czech}   % Alternativní balíček pro sazbu v českém jazyce, kompilovat (pdf)cslatexem!
\graphicspath{ {obrazky/} }
\usepackage{alltt}
\usepackage{amsmath}
\usepackage{listings}
\usepackage{adjustbox}
\usepackage{graphicx}
\usepackage{listings}
\usepackage{hyperref}
\usepackage[most]{tcolorbox}
\usepackage{inconsolata}
\newtcblisting[auto counter]{sexylisting}[2][]{sharp corners, 
    fonttitle=\bfseries, colframe=gray, listing only, 
    listing options={basicstyle=\ttfamily,language=java}, 
    title=Listing \thetcbcounter: #2, #1}
\usepackage[%
%% Z následujících voleb lze použít pouze jednu
% left,               % Rovnice a popisky plovoucich objektů budou %zarovnány vlevo
  center,             % Rovnice a popisky plovoucich objektů budou zarovnány na střed (vychozi)
%% Z následujících voleb lze použít pouze jednu
%semestral						%	sazba zprávy semestrálního projektu
%bachelor						%	sazba bakalářské práce
diploma						 % sazba diplomové práce
%treatise            % sazba pojednání o dizertační práci
%phd                 % sazba dizertační práce
]{thesis}             % Balíček pro sazbu studentských prací
                      % Musí být vložen až jako poslední, aby
                      % ostatní balíčky nepřepisovaly jeho příkazy

%%%%%%%%%%%%%%%%%%%%%%%%%%%%%%%%%%%%%%%%%%%%%%%%%%%%%%%%%%%%%%%%%
%%%%%%      Definice informací o dokumentu             %%%%%%%%%%
%%%%%%%%%%%%%%%%%%%%%%%%%%%%%%%%%%%%%%%%%%%%%%%%%%%%%%%%%%%%%%%%%

%% Název práce:
%  První parametr je název v originálním jazyce,
%  druhý je překlad v angličtině nebo češtině (pokud je originální jazyk angličtina)
\nazev{Multiplatformní aplikace pro správu síťových prvků Mikrotik}{Multiplatform application for Mikrotik network devices management}

%% Jméno a příjmení autora ve tvaru
%  [tituly před jménem]{Křestní}{Příjmení}[tituly za jménem]
\autor[\ Bc.]{Adrián}{Bárdossy}

%% Jméno a příjmení vedoucího včetně titulů
%  [tituly před jménem]{Křestní}{Příjmení}[tituly za jménem]
\vedouci[\ Ing.]{Ondřej}{Krajsa}[Ph.D]

%% Označení oboru studia
% První parametr je obor v originálním jazyce,
% druhý parametr je překlad v angličtině nebo češtině
\oborstudia{Teleinformatika}{Teleinformatics}

%% Označení ústavu
% První parametr je název ústavu v originálním jazyce,
% druhý parametr je překlad v angličtině nebo češtině
\ustav{Ústav telekomunikací}{Department of Telecommunications} 

%% Rok obhajoby
\rok{2018}

%% Místo obhajoby
% Na titulních stránkách bude automaticky vysázeno VELKÝMI písmeny
\misto{Brno}
%% Abstrakt
\abstrakt{
Diplomová práca obsahuje popis vývoju aplikácie na správu sieťových prvkov mikrotik. V úvodnej časti práce sa nachádza popis použitých knižníc, popis API.\\
V ďalšej časti práce sa nachádza naprogramovaná časť backendu aplikácie. Táto časť obsahuje popis jednotlivých zložiek projektu napísaného v programe pycharm. Každá zložka je popísaná jedným súborom a jedným UML diagramom spoločne s tabuľkou metód v triede.\\
V ďalšej časti práce je popis grafickej časti aplikácie, a jej príklad na jednej sekcii naprogramovaných tlačítok. Obsahuje tiež výstup v podobe obrázkov z aplikácie.\\
V poslednej časti práce je návod na nainštalovanie potrebných modulov na spustenie aplikácie a obsahuje tiež manuálne testovanie aplikácie. }
{
Diploma thesis contains the description of the application developement for management of network entities based on mikrotik devices. In the intro, there is the description of used libraries, also description of API.\\
In the next part of thesis there is programmed part of application backend. This part contains description of individual directories of project, which was written in pycharm. Every directory is described by one file together with UML diagram and table of methods in specific class.\\
In the next part of thesis, there is the description of graphical part of the application and its example on one section of programmed buttons. It contains also output in form of pictures from the application.\\
In the last section, there is tutorial for modules instalation, which are needed to run the application and contains manual testing of application.
}

%% Klíčová slova
\klicovaslova{
Mikrotik, API, API-SSL, grafická aplikácia, linux aplikácia, konfigurácia siete, mactelnet, PyQT, python3, tikapy
}{
Mikrotik, API, API-SSL, graphical application, linux application, network configuration, mactelnet, PyQT, python3, tikapy
}

%% Poděkování
\podekovanitext{Rád by som poďakoval vedúcemu bakalárskej práce pánovi Ing.\ Ondřejovi Krajsovi, Ph.D.\ za odborné vedenie, konzultácie, trpezlivosť a podnetné návrhy k~práci.}

%%%%%%%%%%%%%%%%%%%%%%%%%%%%%%%%%%%%%%%%%%%%%%%%%%%%%%%%%%%%%%%%%%%%%%%%

%%%%%%%%%%%%%%%%%%%%%%%%%%%%%%%%%%%%%%%%%%%%%%%%%%%%%%%%%%%%%%%%%%%%%%%%
%%%%%%     Nastavení polí ve Vlastnostech dokumentu PDF      %%%%%%%%%%%
%%%%%%%%%%%%%%%%%%%%%%%%%%%%%%%%%%%%%%%%%%%%%%%%%%%%%%%%%%%%%%%%%%%%%%%%
%% Při vloženém balíčku 'hyperref' lze použít příkaz '\nastavenipdf'
\nastavenipdf
%  Nastavení polí je možné provést také ručně příkazem:
%\hypersetup{
%  pdftitle={Název studentské práce},    	% Pole 'Document Title'
%  pdfauthor={Autor studenstké práce},   	% Pole 'Author'
%  pdfsubject={Typ práce}, 						  	% Pole 'Subject'
%  pdfkeywords={Klíčová slova}           	% Pole 'Keywords'
%}
%%%%%%%%%%%%%%%%%%%%%%%%%%%%%%%%%%%%%%%%%%%%%%%%%%%%%%%%%%%%%%%%%%%%%%%

%%%%%%%%%%%%%%%%%%%%%%%%%%%%%%%%%%%%%%%%%%%%%%%%%%%%%%%%%%%%%%%%%%%%%%%
%%%%%%%%%%%       Začátek dokumentu               %%%%%%%%%%%%%%%%%%%%%
%%%%%%%%%%%%%%%%%%%%%%%%%%%%%%%%%%%%%%%%%%%%%%%%%%%%%%%%%%%%%%%%%%%%%%%
\begin{document}


%% Vložení desek generovaných informačním systémem
\includepdf[pages=1,offset=19mm 0mm]%
  {pdf/student-desky}% název souboru nesmí obsahovat mezery!
% nebo vytvoření desek z balíčku
%\vytvorobalku
\setcounter{page}{1} %resetovani citace stranek - desky se necisluji

%% Vložení titulního listu generovaného informačním systémem
\includepdf[pages=1,offset=19mm 0mm]%
  {pdf/student-titulka}% název souboru nesmí obsahovat mezery!
% nebo vytvoření titulní stránky z balíčku
%\vytvortitulku
   
%% Vložení zadání generovaného informačním systémem
\includepdf[pages=1,offset=19mm 0mm]%
  {pdf/student-zadani}% název souboru nesmí obsahovat mezery!
% nebo lze vytvořit prázdný list příkazem ze šablony
%\stranka{}%
%	{\sffamily\Huge\centering ZDE VLOŽIT LIST ZADÁNÍ}%
%	{\sffamily\centering Z~důvodu správného číslování stránek}

%% Vysázení stránky s abstraktem
\vytvorabstrakt

%% Vysázení prohlaseni o samostatnosti
\vytvorprohlaseni

%% Vysázení poděkování
\vytvorpodekovani

%% Vysázení poděkování projektu SIX
% ----------- zakomentujte pokud neodpovida realite
\vytvorpodekovaniSIX

%% Vysázení obsahu
\obsah

%% Vysázení seznamu obrázků
\seznamobrazku

%% Vysázení seznamu tabulek
\seznamtabulek

%% Vložení souboru 'text/uvod.tex' s úvodem
\chapter{Úvod do diplomovej práce}
Diplomová práca na tému "Multiplatformní aplikace pro správu síťových prvku Mikrotik" sa bude primárne zaoberať samostatným mikrotikom. Primárne pomocou aplication programable interface (API) vytvorenie jej konzolovej časti (backendu) a grafickej časti (frontendu). Tieto dve časti dajú celkovú applikáciu dokopy ako celok.\\
V prvej časti práci bude definovanie Mikrotik API a jeho možností, porovnanie podobnosti s operačným systémom unix. Ďalej budú popísané možnosti zabezpečenia API pomocou secure socket layer (SSL). Budú tu tiež spomenuté použité porty,a ďalšie možnosti.\\
V druhej časti práce bude popis API a spôsoboch softvérového riešenia aplikácie pre správu Mikrotikov. Táto časť bude tiež obsahovať niečo ohľadom technológie git, popise, čo je git, princíp tzv. commitu a pushu. Rozdiely medzi vetvami, prepínanie medzi vetvami a pridávania zmien. Taktiež tu bude spomenutý aj úvod do certifikátov a to konkrétne Single Sign-on metódy.\\
V ďalšej časti bude návrh riešenia softvérovej implementácie aplikácie. Bude obsahovať popis, princípy, diagramy, hlavne Unified modeling language (UML), popisy knižníc, jednotlivých tried a modulov. Každý modul bude popísaný svojou funkcionalitou, parametrami a výstupom s praktickými ukážkami.\\
V ďalšej časti bude použitá implementácia softvérového návrhu riešenia. Bude tu riešenie ako v konzolovej časti, jeho ukážky, test a výsledky. \\
V poslednej časti práce bude ukážka grafického spracovania konzolovej časti aplikácie a ich prepojenia do jednej aplikácie, spoločne s ukážkami kódov, testu  a výsledkov.
\chapter{Mikrotik a RouterOS (SwitchOS)}
V dnešných malých  a stredne veľkých firmách sa na správu siete používajú prevažne routre a switche typu Mikrotik. Mikrotik je firma vyvíjajúca routre a switche,prístupové body a ďalšie sieťové prvky v Litve. \\
Mikrotik zariadenia používajú operačný systém routerOS, prípadne switchOS. Rozdiel medzi nimi je na základe použitého zariadenia. Čo sa týka routrov, používa operačný systém routerOS, switch používa switchOS, v prípade prístupových bodov (AP) je to routerOS. 
\section{Mikrotik API}
Za pomoci Mikrotik API môžeme programovať užívateľské programy a prostredia na riadenie a konfiguráciu Mikrotik zariadení. V dnešnej dobe existuje softvér na konfiguráciu mikrotik zariadení a to pod názvom \textbf{Winbox}. Winbox v dnešnej dobe existuje len na operačný systém Windows a Macintosh (MAC). Bohužiaľ na operačný systém Linux winbox samostatne neexistuje a musí sa simulovať pomocou emulátoru Windows aplikácií za pomoci programu Wine. Toto spôsobuje komplikácie pri použití niektorých funkcií winboxu ale aj iných programov operačného systému Windows. Výstupom práce bude práve Graphical User Interface (GUI). 
\subsection{Požiadavky na použitie API}
\begin{itemize}
\item Verzia routerOS verzie \textit{3.0.X} a vyššie \cite{API}
\end{itemize}
\subsection{Porty}
Základné porty na použitie Mikrotik API \cite{API} sú:
\begin{itemize}
\item \textbf{API port}: 8728
\item \textbf{Application programable interface Secure Socket Layer (API-SSL) port}: 8729
\end{itemize}
\subsection{Základný port 8728}
Na základné pripojenie k API aplikácii na prvku Mikrotik musí byť povolený port 8728, ktorý tiež nájdeme v IP-> Services spoločne s API-SSL.\\
Na základné pripojenie nie je potreba žiadneho transport layer security (TLS) certifikátu. Stačí jednoducho napísať kód a skompilovať ho. 
\subsection{SSL port 8729}
Pre použitie portu 8729 tiež známeho ako API-SSL portu je potreba zabezpečenej komunikácie pomocou SSL protokolu. \\
Primárne muisú byť natavený port, základný port 8729 v IP -> Services. Môžeme ale definovať aj užívateľsky definovaný port. \\
Možnosti nastavenia API-SSL:
\begin{itemize}
\item prístup bez certifikátu TLS
\item prístup pomocou certifikátu TLS
\end{itemize} 
\subsubsection{Prístup pomocou certifikátu TLS}
Pre použitie certifikátu TLS je potrebné vygenerovať certifikát TLS, a to na certifikačnej autorite alebo na ľubovoľnej linux stanici ideálne, ale tiež to dokážeme spraviť aj na WIndows stanici či MAC. 
Spôsoby vygenerovania certifikátov:
\begin{itemize}
\item openssl
\item easy-rsa 
\item Windows Server Certificate Services
\end{itemize}
\subsubsection{Openssl}
Openssl \cite{OpenSSL} je softvér na generovanie certifikátov pre komunikáciu v počítačovej sieti. Koreňovo sa používa na prístup na web skrz protokol Hyper Trasfer Transport Protocol Secure (HTTPS). Pre vygenerovanie certifikátov sa musí vygenerovať: \begin{itemize}
\item certifikát \textit{*.crt}
\item certifikačný požiadavok \textit{*.csr}
\item kľúč k certifikátu \textit{*.key}
\end{itemize}
\subsubsection{Easy-rsa}
Softvér easy-rsa \cite{EasyRSA} sa používa na vytvorenie open-source certifikačnej autority a užívateľých certifikátov napr. pre potreby HTTPS spojenia.\\
Po nainštalovaní easy-rsa napr. na Ubuntu príkazom \textit{sudo apt install easy-rsa} sa musí spraviť nasledovné: \begin{itemize}
\item Nakopírovanie konfiguračných súborov do zložky autority
\item Vytvorenie šablóny na vygenerovanie certifikačnej autority
\item Vytvorenie užívateľksých certifikátov
\end{itemize}
\subsubsection{Active Directory Certificate Services}
Windows riešenie \cite{WindowsCA} pre generovanie  certifikačnej autority je inštalácia roly servera Active DIrectory Certificate Services. \\
Pre použitie certifikačnej autority na Windows servery je potreba:
\begin{itemize}
\item Inštalácia role serveru
\item Nadefinovanie certifkačnej autority
\item Generovanie certifikátov
\end{itemize}
\section{API slová}
\label{chap:APIwords}
API slová \cite{API} sú základnou časťou API "vety". API "veta" predstavuje príkaz v pouužití príkazu napr. \textit{\//ip/address/print}, \textit{\//ip/address/add address="10.1.1.1/24" interface="ether1"}. \\
Parametre na slová:
\begin{itemize}
\item každé slovo má svoju zakódovanú dĺžku t.j. 
\begin{itemize}
\item 0 - 127 bitov zaberá 1 Byte
\item 128 - 1023 bitov zaberá 2 Byty 
\item 1024 bitov - 2097 kib zaberá 3 Byty
\item viac ako 2098 kib zaberá 4 Byty
\end{itemize}
\item jednotlivé slová súzaradené do viet
\item maximum bztov na slovo sú 4 Byty
\item kontrolné byty sa nepoužívajú
\end{itemize}
\section{Príkazové slová API}
Slová Mikrotik API sa zaraďujú do API viet použitím API slov, na ktoré platia požiadavky, ktoré sú spomenuté v kapitole \ref{chap:APIwords}.Na použitie API viet je potreba začínať znakom \textit{\//}. Napr. miesto \textit{ip address print} sa použije \textit{\//ip/address/print}.\\
Pre úplnosť API viet musí platiť \cite{API}
\begin{itemize}
\item zakódovaná dĺžka slova
\item slovo musí začínať znakom \//
\item musí byť použitá správna syntax
\end{itemize}
\section{Použitie atribútov v príkaze a filtrovanie}
V prípade konfigurácie mikrotik zariadení sa pre nastavenie jednotlivých prvkov používajú tzv. atribúty \cite{API} napr. ip adresa, číslo pravidla, meno rozhrania, nastavenie virtuálnej lokálnej sieti (VLAN). \\
Použitie atribútov má špeciálnu syntax pre konfiguráciu prípadne zmenu prvku na mikrotiku, prípadne pridanie a zmazanie prvku. Na použitie atribútov sa použije špeciálny znak \textit{=}. Napr. \textit{\//ip/address/add =address=10.1.1.1/24 =interface=erher1}.\\
Pre filtrovanie prvkov v rámci mikrotik API syntaxe sa používa špeciálny atribút parameter so znakom \textit{?}. Napr. \//ip/address/print =?type=ether1 vyfiltruje len rozhranie ether1.
\section{Špeciálne slová API}
Miktotik API má možnosť tzv. špeciálnych slov \cite{API}. Špeciálne slová sú slová, ktoré sú rezervované  a nesmú sa použiť pre iné použitie ako napríklad meno premennej, metódy, triedy, a iné. Medzi špeciálne slová patria:\begin{itemize}
\item prihlásenie sa na zariadenie \//login
\item ukončenie spojenia na zariadenie \//cancel
\item odhlásenie sa zo zariadenie \//logout
\item získanie všetkých parametrov \//getall
\end{itemize}   
\chapter{Pripojenie na Mikrotik}
\section{Možnosti pripojenia}
Pripojenie na mikrotik je realiyované pomocou niekoľkých typov softvéru:\begin{itemize}
\item \textbf{Winbox} - základný softvér na konfiguráciu mikrotiku
\item \textbf{Webfig} - konfigurácia mikrotiku pomocou webového rozhrania štandardne na portoch 80 a 443
\item Riadenie mikrotiku pripojením na mac adresu - \textbf{mactelnet}
\item Pripojenie pomocou protokolu \textbf{SSH} - zabezpečené a šifrované spojenie
\item Pripojenie pomocou protokolu \textbf{telnet} - nebezpečné v dnešnej dobe
\end{itemize}
\section{Pripojenie pomocou winboxu}
Winbox\cite{winbox} je nástroj na administráciu mikrotiku. Medzi jeho vlastnosti patrí:\begin{itemize}
\item GUI nástroj (klikátko)
\item rýchlosť
\item spoľahlivosť 
\end{itemize} 
\\Winbox je prepis konzolovej aplikácie do grafickej. Obsahuje tiež nástroje ktoré sa v konzole nedajú odsimulovaťnapr. graphs, torch, netmon, scheduler,...\\
Niektoré funkcie nevieme meniť pomocou winboxu napr. Media Access Control (MAC) adresu rozhrania. 
\begin{figure}[H]
\centering
\includegraphics[scale=0.2]{../text/winbox.png}
\caption{Winbox základné prihlasovacie rozhranie}
\label{fig:winbox}
\end{figure} 
Režimy winboxu:\begin{itemize}
\item jednoduchý režim - obsahuje na pripojenie len užívateľské meno, heslo a adresu mikrotiku
\item pokročilý režim - možnosť pridania skupiny mikrotikov, popisky a názov spojenia
\end{itemize}
\section{Pripojenie pomocou webfigu}
Webfig\cite{webfig} je webová aplikácia RouterOS a umožňuje konfiguráciu, minitoring a údržbu prvkov RouterOS. Medzi hlavné tasky webfigu patrí:\begin{itemize}
\item konfigurácia mikrotiku
\item mnotring mikrotiku
\item riešenie problémov na mikrotiku za pomoci webového rozhrania
\end{itemize}
\begin{figure}[H]
\centering
\includegraphics[scale=0.4]{../text/webfig.png}
\caption{Webfig základné prihlasovacie rozhranie}
\label{fig:webfig}
\end{figure} 
\section{Mactelnet}
Mactelnet\cite{mactelnet} predstavuje aplikačný protokol riadený na druhej vsrtve referenčného modelu. Tiež predtavuje kombináciu winboxu  a telnetu v jednom protokole. Riadi prístup na napr. nový mikrotik, ktorý ešte neobsahuje žiadnu konfiguráciu. Pracuje absolútne rovnakým spôsobom ako telnet. Je možné sa pripojiť len na fyzicky pripojený mikrotik pomocou mactelnet, vzdialený prístup pomocou mactelnet nie je možný. 
\begin{figure}[H]
\centering
\includegraphics[scale=0.4]{../text/mactelnet.png}
\caption{Výstup príkazu mactelnet}
\label{fig:webfig}
\end{figure}
Po pripojení na mikrotik pomocou mactelnet sa nastaví základná konektivita a pripájame sa potom na základe Internet Protocol (IP) adresy. 
\section{Pripojenie pomocou telnet a SSH}
Ďalšou možnosťou pripojenia na mikrotik je prihlásenie sa pomocou telnetu\cite{telnet} prípadne SSH\cite{ssh} na konzolu mikrotiku. Napríklad na nastavenie fronty,firewallu,... .
\subsection{Pripojenie cet telnet}
Telnet predstavuje protokol, ktorý umožňuje pripojenie na vzdialené servery. Jeho štandardným portom je port 23. \\
Na povolenie pripojenia pomocou telnetu je potrebné povoliť službu telnet na mikrotiku v IP -> Services. Pre bezpečnostné účely by sa telnet nemal používať, je terčom útokov nakoľko je nešifrovaný. Pokiaľ chceme povoliť telnet na pripojenie na mikrotik, by sa mal minimálne zmeniť štandardný port z 23 na užívateľsky definovaný port.\\
Príklad príkazu na pripojenie na zariadenie pomocou telnetu: \textit{telnet <IP adresa> <port>} 
\subsection{Pripojenie pomocou ssh} 
SSH predstavuje protokol, ktorý umožňuje vzdialené pripojenie pomocou tohoto protokolu. Používa štandardný port 22. Tak isto ako u telnetu, pre SSH platí to isté, je potrebné ho povoliť v IP -> Services. SSH na rozdiel od telnetu je ale šifrovaný  a zabezpečený protokol. SSH predstavuje bezpečnú verziu telnetu. Je možné si zabezpečiť SSH prístup na bezpečnejší, a to tak, že sa budú porovnávať verejný  a súkromný kľúč certifikátu TKIP. V7stupom pripojenia SSH na mikrotik je na obrázku \ref{fig:ssh}. 
\begin{figure}[H]
\centering
\includegraphics[scale=0.4]{../text/ssh.png}
\caption{Prihlásenie na mikrotik pomocou príkazu SSH}
\label{fig:ssh}
\end{figure}
\chapter{Programovací jazyk Python}
Python je interpretovaný, interaktívny, objektovo-orientovaný a vysoko-úrovňový programovací jazyk.Jazyk Python bol vytvorený pánom  Guido van Rossum v Wiskundskom centre informatiky v 80-tych rokoch. \\
Medzi jeho vlastnosti patrí:\begin{itemize}
\item dynamické typovanie
\item konzolové aplikácie
\item objektové aplikácie
\item všetko v pythone je objekt
\item jednoduchá syntax
\item biele znaky sú súčasťou jazyka
\item dynamické typy premenných
\item široká škála knižníc
\item dokumentácia na vysokej úrovni\
\item používaný  na webové aplikácie, strojové učenie, teórie zložitosti,...
\end{itemize}
Verzie jazyku Python:\begin{itemize}
\item Python verzia 2
\item Python verzia 3
\end{itemize}
\section{Python 2}
Vlastnosti jazyku Python 2\cite{Python2}
\begin{itemize}
\item automatická spáva pamäti (garbage collector)
\item podporuje viac vstupných paradigiem
\item Volanie niektorých príkazov je odlišné od Python 3
\item referenčný interpret sa nazýva CPython a spravuje ho organizácia Python Software Foundation
\item Súčasne sa používa Python vo verzii 2.7.2
\end{itemize}
\section{Python 3}
Vlastnosti jazyku Python 3\cite{Python3}\begin{itemize}
\item V niekorých častiach syntaxe v porovnaní s jazykom Python 2je trošku odlišná (napr. príkaz print,...)
\item Od verzie 3.6 má premenná typu slovník interné zachovávané poradie vkladaných prvkov
\item Pridanie anotácií cez metatriedy
\item deklarácia nelokálnej premennej vonku z funkcie
\item Slová typu True, False a None sú rezervované slová
\item Mnoho vlastností ma rovnakých ako Python 2
\item Miesto <> sa voverzii 3 používa relačný operátor !=
\item od Júla 2018 by mala výjsť verzia Python 3.7 s ďalšími novinkami 
\end{itemize}
\section{Prostredia na programovanie v jazyku Python}
Na realizovanie python programu je nutnosť mať nainštalovaný softvér na komppiláciu softvéru napísaného v jazyku Python. Na tieto účely slúži tzv. intergrated developement envinroment (IDE). Ecistuje niekoľko ciet aj mimo IDE ako spustiť kód napísaný v jazyku python.\begin{itemize}
\item Napísanie kódu napr. v textovom editore typu nano, vim, gedit ale aj windows riešenie ako napr. notepad
\item nainštalovaný python kompilátor
\item spustenie programu príkazom python <názov.py>
\end{itemize}
\section{Pycharm}
Pycharm predsatvuje IDE na pokročilé aplikácie napísané v jazyku Python. Exustuje v dcoch verziách:\begin{itemize}
\item Pycharm Community Edition - voľne dostupné, nelicencované, neobsahuje niektoré doplnky professional verzie
\item Pycharm Professional Edition - licencované, voľne dostupné na 30 dní, licencované, plný prístup ku všetkým doplnkom
\end{itemize}
Na nainštalovanie pycharm ľubovoľnej verzie je potreba:\begin{itemize}
\item Stiahnutie bin respektíve exe súboru inštalátoru
\item Napojenie na pracovný adresár projektu
\item Napojenie na tzv. envinroment, to je použitie cesty k volaniu príkazu python respektíve python3
\end{itemize}
Po spustení programu Pycharm sa vytvorí projekt, kde po jeho inicializácii nájdeme podobný výstup.
\begin{figure}[H]
\centering
\includegraphics[scale=0.25]{../text/pycharm.png}
\caption{Rozhranie IDE Pycharm Professional Edition}
\label{fig:webfig}
\end{figure} 
Jednou z najväčších výhod je generovanie UML diagramov z kódu.
\chapter{Použité knižnice v diplomovej práci}
\section{OS.SYSTEM}
\section{Telnetlib}
\section{Pxssh}
\section{TikApy}






%% Vložení souboru 'text/reseni' s popisem reseni práce
\chapter{Konzolová časť aplikácie na správu mikrotikov}
V tejto kapitole si popíšeme fungovanie naprogramovanej aplikácie. Celkovo je konzolová časť aplikácie napísaná za pomoci knižnice tikapy popísanej v kapitole \ref{sec:tikapy}. Kapitola bude rozdelená do niekoľkých častí:\begin{itemize}
\item časť 1: popis naprogramovanej časti pre vyhľadávanie mikrotikov, pripojenie sa na mikrotik cez python pomocou protokolov telnet, SSH, mactelnet a napojenie na metódy
\item časť 2: Popis infraštruktúry backendu - zložky, ich vysvetlenie, zoznam súborov na konfiguráciu mikrotku, vysvetenie rozdelenia, vysvetlenie tried, metód daných tried a volanie funkcií
\item časť 3 - prtidanie tabuliek jednotlivých tried a ich metód v každej zložke, krátka sumarizácia, ich niektoré vybrané UML diagramy, ostatné budú zahrnuté v prílohe
\end{itemize}
\section{Popis naprogramovanej časti prihlasovania na mikrotik}
\label{sec:popis1}
V tejto časti si zobrazíme rozbor časti prihlasovania na mikrotik a základné funnkcie. Toto je riadené v rámci projektu nazvaného \textit{diplomkap3} v ktorom je súbor \textit{loginManager.py}. V rámci login managera tu nachádzame:
\section{Rozbor hlavnej časti backendu}
V rámci hlavnej konfiguračnej časti diplomovej práce, pre konfiguráciu backendu mikrotiku za pomoci porgramovacieho jazyka python som projekt rozdelil do niekoľkých častí:\begin{itemize}
\item \textbf{bridge} - táto časť obsahuje prvky konfiguácie, pridania, odstránenia, zapnutia, vypnutia možnosti bridgu na mikrotiku, konfigurácia existujúceho bridgu, zobrazenie zoznamu bridgov
\item  \textbf{capsman} - táto časť obsahuje konfiguráciu hromadnej obsluhy mikrotik úrístupových bodov a WiFi, profily, bezpečnosť, konfiguácie, povolené rýchlosti, zobrazenie zoznamu pripojených prvkov a ďalšie funkcie
\item \textbf{certs} - obsahuje certifikáty na pripojenie sa na mikrotik pomocou protokolu api-ssl
\item \textbf{Dude} - obsahuje popis konfigurácie ako nastaviť nástroj Dude klienta, ako nakonfigurovať Dude na vzdialený monitoring na Dude serveri, taktiež Dude server, a ďalšie možnosti
\item \textbf{exportToHtml} - časť predstavuje generovanie súboru na analýzu v podobe webovej stránky
\item \textbf{interfaces} - časť predstavuje konfiguráciu rozhraní na mikrotiku, tieto časti sú tiež popísané aj v iných zložkách ako napr. bridge. 4asť popisuje pridanie, odstránenie, zapnutie, vypnutie, konfiguráciu existujúcich rozhraní.
\item \textbf{IPv4} - rozsiahla časť, obsahuje konfiguráciu IP adries, firewallu, monitoringu, smerovania a ďalších nástrojov spadajúcich pod IP zložku na mikrotiku.
\item \textbf{IPv6} - pre zložku IPv6 platí to isté čo pre zložku IPv4, ale platí pre konfiguráciu na základe IPv6 adresného rozsahu
\item \textbf{KVM} - sekcia bude popisovať možnosti virtualizácie mikrotiku.
\item \textbf{log} - sekcia bude popisovať analýzu a konfiguráciu logu zariadenia
\item \textbf{makeSupportFile} - seckia bude popisovať vytvorenie súboru potrebného pre analýzu na mikrotik podpore
\item \textbf{mesh} - sekcia popisuje konfiguráciu tzv. mesh technológie, technológii podobne ako  v rámci časti bridge
\item \textbf{MPLS} - sekcia bude popisovať možnosti konfiurácie Multi Protocol Label Switching (MPLS), jej pridanie, odstránenie ,zapnutie, vypnutie, modifikácie a ďalšie funkcie.
\item \textbf{PPP} - sekcia bude popisovať konfiguráciu Point to Point Protocol (PPP) a ďalších možností Virtual Private Network (VPN) konfigurácie.
\item \textbf{Queues} - sekcia budep popisovať konfiguráciu sieťových front, možnosti front, typy front a ďalšie funkcie
\item \textbf{Radius} - sekcia bude popisovať nastavenie funkcie Radius - autentizačnej služby užívateľov , jeho modifikáciu, konfiguráciu a ďalšie funkcie.
\item \textbf{Routing} - sekcia bude popisovať možnosti dynamického smerovania, statické smerovanie bude popísané v rámci časti IPv4, dynamické smerovacie protokoly, ich konfigurácie, a ďalšie možnosti. 
\item \textbf{Switch} - sekcia bude popisovať konfiguráciu prepínača, niektoré mikrotiky sú typu SwitchOS a sú štandardne prepínač. Kofiguuráciu portov, trunkov, a ďalších funkcií. 
\item \textbf{System} - sekcia bude popisovať časť konfigurácie systémových nástrojov, ich funkcií a konfigurácie, a ďalších funkcií.
\item \textbf{Tools} - sekcia bude popisovať konfiguráciu mikrotik nástroj, a však nie všetky bolo možné odsimulovať v rámci konzolvej časti aplikácie, ich konfiguráciu, spustenie, riadenie a ďalšie funkcie. 
\item \textbf{Wireless} - sekcia bude obsahovať konfiguráciu bezdrátového rozhrania, moduly, módy, konfiguráciu, nastavenie, a ďalšie funkcie
\item \textbf{konfiguračné súbory mimo zložiek} - sekcia popísaná v kapitole  \ref{sec:popis1}, popisuje súbory na základnú konfiuráciu mikrotiku, nastavenie základnej konfigurácie.
\end{itemize}
Ukážka súborovej štruktúry je zobrazená na obrázku \ref{fig:structure}:
\begin{figure}[H]
\centering
\includegraphics[scale=0.4]{../text/struktura.png}
\caption{Štruktúra projektu konzolovej časti projektu}
\label{fig:structure}
\end{figure} 

%% Vložení souboru 'text/vysledky' s popisem vysledků práce
\chapter{Grafická časť aplikácie (frontend)}
V rámci frontend časti aplikácie boli použité aplikácie:\begin{itemize}
\item \textbf{QT4 designer} - softvér na nábrh dizajnu "okien" aplikácie
\item \textbf{PyQT 4} - doplnok do pythonu, na návrh a programovenie grafických aplikácií
\end{itemize}
Všetky implementácie v kóde boli použité vo verzovacom systéme github za použitia technológie git. Technológia git predstavuje verzovací systém na rôzne vetvy, hlavnú vetvu predstavuje master vetva, potom sú tu užívateľsky definované vetvy. \\
Git pracuje na základe \textit{repozitárového systému}. Postup pridania súborov na git repozitár:\begin{itemize}
\item inicializácia git repozitáru - príkaz \textit{git init}
\item pridanie súborov aktuálnej zložky - príkaz \textit{git add .}
\item commit na repozitár - príkaz \textit{git commit -m "Text"}
\item pridanie súborov na git - príkaz \textit{git push origin vetva}
\end{itemize} 
\section{QT4 disigner}
\section{PyQT 4}
\section{Aplikovanie v aplikácii}


%% Vložení souboru 'text/zaver' se závěrem
\chapter{Záver}
Diplomová práca sa zaoberala vytvorením užívateľského rozhrania pre prvky mikrotik na rôzne operačné systémy. Primárne funguje na operačný systém Linux a macOS, ale z časti funguje aj na operačný systém Windows, len je potreba doinštalovať konkrétne knižnice a moduly.\\
Práca obsahuje v prvej časti teoretický rozbor API, knižníc a pythonu. Popisuje možnosti inštalácie, princíp fungovania modulov a použité knižnice.\\
V ďalšej časti sa nachádza popis konzolovej časti aplikácie spôsobom výberu jedneého súboru zo zložky v projekte a popisom jeho metód  a reprezentáciou jeho UML diagramu. Pre každú zložku je najskôr globálny popis a následne vybratý jeden súbor zo zložky a ten je detajlne popísaný.\\
V ďalšej časti  sa nachádza grafická časť aplikácie (frontend) realizovaná cez PyQT4 a QT4 designer. Nachádza sa tu popis prostredia, možnosti, výnimky. Ďalej sú tu v skratke popísané funkcie tlačítok a jeho súbory sú uložené na priloženom DVD a na verzovacom systíme github s konkrétnymi repozitármi.\\
V poslednej časti  práce sa nachádzajú návody na inštaláciu na UNIX systémoch  a na Windows systémoch. Ako posledná časť slúži tetsovanie aplikácie. \\
Práca splňuje zadanie a má vytvorené užívateľské rozhranie,  v rámci API je tam možnosť ďalšej práce. Aplikácia predstavuje konfiguračný nástroj na konfiguráciu prvkov mikrotik, ale vzhľad aplikácie je patrične odlišný od originálneho winboxu. Obsahuje systémové nástroje, nastavenie IP a Systémových nástrojov, vzhľad aplikácie je poňatý ale iným spôsobom ako to je vo winboxe. 

%% Vložení souboru 'text/literatura' se seznamem literatury
\begin{literatura}{99}

\bibitem{API}
      \emph{Manual:API}\/ [online].
    2014,  [cit.\,24.\,03.\,2018].
    Dostupné z~URL:
    \(<\)\url{https://wiki.mikrotik.com/wiki/Manual:API}\(>\)			
    
  \bibitem{EasyRSA}
      \emph{How to Install & Configure Easy-RSA}\/ [online].
    2013,  [cit.\,24.\,03.\,2018].
    Dostupné z~URL:
    \(<\)\url{https://docs.bigchaindb.com/projects/server/en/latest/production-deployment-template/easy-rsa.html}\(>\)

\bibitem{mactelnet}
      \emph{MAC Level Access (Telnet and Winbox)}\/ [online].
    2007,  [cit.\,26.\,03.\,2018].
    Dostupné z~URL:
    \(<\)\url{https://mikrotik.com/testdocs/ros/2.9/tools/mactelnet.php}\(>\)

\bibitem{OpenSSL}
     Mitchell Anicas \emph{OpenSSL Essentials: Working with SSL Certificates, Private Keys and CSRs}\/ [online].
    2012,  [cit.\,24.\,03.\,2018].
    Dostupné z~URL:
    \(<\)\url{https://www.digitalocean.com/community/tutorials/openssl-essentials-working-with-ssl-certificates-private-keys-and-csrs}\(>\)		

\bibitem{OS}
      \emph{os — Miscellaneous operating system interfaces}\/ [online].
    2012,  [cit.\,04.\,04.\,2018].
    Dostupné z~URL:
    \(<\)\url{https://docs.python.org/2/library/os.html}\(>\)		

\bibitem{OSPF}
    \emph{CISCO: Open Shortest Path First (OSPF)}\/ [online].
    2009,  [cit.\,09.\,11.\,2014].
    Dostupné z~URL:
    \(<\)\url{http://www.cisco.com/c/en/us/products/ios-nx-os-software/open-shortest-path-first-ospf/index.html}\(>\).	

\bibitem{pexpect}
    \emph{Pexpect version 4.4}\/ [online].
    2018,  [cit.\,02.\,04.\,2018].
    Dostupné z~URL:
    \(<\)\url{https://pexpect.readthedocs.io/en/stable/}\(>\).	
    
\bibitem{pexpectinstall}
    \emph{Installation}\/ [online].
    2018,  [cit.\,02.\,04.\,2018].
    Dostupné z~URL:
    \(<\)\url{https://pexpect.readthedocs.io/en/stable/install.html}\(>\).	
    
\bibitem{pxssh}
    \emph{pxssh (version 2.3)}\/ [online].
    2018,  [cit.\,02.\,04.\,2018].
    Dostupné z~URL:
    \(<\)\url{http://pexpect.sourceforge.net/pxssh.html}\(>\).

\bibitem{Python2}
    \emph{Python 2.7.2 Release}\/ [online].
    2018,  [cit.\,02.\,04.\,2018].
    Dostupné z~URL:
    \(<\)\url{https://www.python.org/download/releases/2.7.2/}\(>\).	
        
\bibitem{ssh}
     Tatu Ylonen \emph{SSH PROTOCOL)}\/ [online].
    2017,  [cit.\,26.\,03.\,2018].
    Dostupné z~URL:
    \(<\)\url{https://www.ssh.com/ssh/protocol/}\(>\)

\bibitem{telnet}
      \emph{What is telnet?)}\/ [online].
    2018,  [cit.\,26.\,03.\,2018].
    Dostupné z~URL:
    \(<\)\url{https://kb.iu.edu/d/aayd}\(>\)
    
\bibitem{telnetlib}
      \emph{telnetlib — Telnet client)}\/ [online].
    2018,  [cit.\,04.\,04.\,2018].
    Dostupné z~URL:
    \(<\)\url{https://docs.python.org/3.1/library/telnetlib.html}\(>\)

\bibitem{webfig}
     \emph{Manual: Webfig}\/ [online].
    2018,  [cit.\,26.\,03.\,2018].
    Dostupné z~URL:
    \(<\)\url{https://wiki.mikrotik.com/wiki/Manual:Webfig}\(>\)	

\bibitem{Python3}
    \emph{What’s New In Python 3.0}\/ [online].
    2018,  [cit.\,02.\,04.\,2018].
    Dostupné z~URL:
    \(<\)\url{https://docs.python.org/3.0/whatsnew/3.0.html}\(>\).

\bibitem{tikapy}
    \emph{tikapy}\/ [online].
    2018,  [cit.\,02.\,04.\,2018].
    Dostupné z~URL:
    \(<\)\url{https://github.com/vshn/tikapy/blob/master/README.rst}\(>\).

\bibitem{winbox}
     \emph{Manual: Winbox}\/ [online].
    2018,  [cit.\,26.\,03.\,2018].
    Dostupné z~URL:
    \(<\)\url{https://wiki.mikrotik.com/wiki/Manual:Winbox}\(>\)		
    
 \bibitem{WindowsCA}
      \emph{Install the Certification Authority}\/ [online].
    2017,  [cit.\,24.\,03.\,2018].
    Dostupné z~URL:
    \(<\)\url{https://docs.microsoft.com/en-us/windows-server/networking/core-network-guide/cncg/server-certs/install-the-certification-authority}\(>\)
    

%% Vložení souboru 'text/zkratky' se seznam použitých symbolů, veličin a zkratek
\begin{seznamzkratek}{90}
	
	\novazkratka{zkAP}		% název
		{AP}								% zkratka
		{Prístupový bod}	
	
	\novazkratka{zkAPI}		% název
		{API}								% zkratka
		{Application programable interface}
	
	
	\novazkratka{zkAPI-SSL}		% název
		{API-SSL}								% zkratka
		{Application programable interface Secure Socket Layer}	
	
	\novazkratka{zkFTP}		% název
		{FTP}								% zkratka
		{File Transfer Protocol}		
	
	\novazkratka{zkGUI}		% název
		{GUI}								% zkratka
		{Graphical User Interface}	
	
	\novazkratka{zkIDE}		% název
		{IDE}								% zkratka
		{Integrated Developement Envinroment}		
	
	\novazkratka{zkIP}		% název
		{IP}								% zkratka
		{Internet Protocol}	
	
	\novazkratka{zkIPSEC}		% název
		{IPSEC}								% zkratka
		{Internet Protocol Security}
	
	\novazkratka{zkMAC}		% název
		{MAC}								% zkratka
		{macintosh}
		
	\novazkratka{zkMAC}		% název
		{MAC}								% zkratka
		{Media Access Control}	
	
	\novazkratka{zkMPLS}		% název
		{MPLS}								% zkratka
		{Multi Protocol Label Swiching}		
	
	\novazkratka{zkOS}		% název
		{OS}								% zkratka
		{Operačný systém}		
	
	\novazkratka{zkPPP}		% název
		{PPP}								% zkratka
		{Point to Point Protocol}		
	
	\novazkratka{zkSSL}		% název
		{SSL}								% zkratka
		{Secure Socket Layer}
	
	\novazkratka{zkTLS}		% název
		{TLS}								% zkratka
		{Transport Layer Security}	
	
	\novazkratka{zkUML}		% název
		{UML}								% zkratka
		{Unified Modeling Language}
	
	\novazkratka{zkVPN}		% název
		{VPN}								% zkratka
		{Virtual Private Network}	
												
\end{seznamzkratek}


%% Začátek příloh
\prilohy

%% Vysázení seznamu příloh
\seznampriloh

%% Vložení souboru 'text/prilohy' s přílohami
\section{Uložené git repozitáre}
Obsah súborov na projekt aplikácie nájdeme na repozitári github: \href{//github.com/adiosko/mikrotik/tree/adrian}{mikrotik}\\
Diplomová práca v textovej podobe je uložená na repozitári github: \href{https://github.com/adiosko/tex_diplomka_mikrotik/tree/adrian}{diplomkatex}
\section{Obsah priloženého DVD}
DVD obsahuje súbory projektu uložené v zložkách:
\begin{itemize}
\item \textbf{bridge} - táto časť obsahuje konfiguráciu bridgu
\item  \textbf{capsman} - táto časť obsahuje konfiguráciu centrálnej obsluhy mikrotik prístupových bodov  ďalšie funkcie
\item \textbf{certs} - obsahuje certifikáty na pripojenie sa na mikrotik pomocou protokolu API-SSL
\item \textbf{Dude} - obsahuje konfiguráciu Dude
\item \textbf{exportToHtml} - časť predstavuje generovanie súboru na analýzu v podobe webovej stránky
\item \textbf{interfaces} - časť predstavuje konfiguráciu rozhraní na mikrotiku
\item \textbf{IPv4} - rozsiahla časť, obsahuje konfiguráciu IP adries, ...
\item \textbf{IPv6} - pre zložku IPv6 platí to isté čo pre zložku IPv4, ale platí pre konfiguráciu na základe IPv6 adresného rozsahu
\item \textbf{KVM} - sekcia obsahuje možnosti virtualizácie mikrotiku
\item \textbf{log} - sekcia obsahuje analýzu a konfiguráciu logu zariadenia
\item \textbf{makeSupportFile} - seckia obsahuje vytvorenie súboru potrebného pre analýzu na mikrotik podpore
\item \textbf{mesh} - sekcia popisuje konfiguráciu tzv. mesh technológie
\item \textbf{MPLS} - sekcia obsahuje možnosti konfiurácie Multi Protocol Label Switching (MPLS)
\item \textbf{PPP} - sekcia obsahuje konfiguráciu Point to Point Protocol (PPP) a ďalších možností Virtual Private Network (VPN) konfigurácie.
\item \textbf{Queues} - sekcia obsahuje konfiguráciu sieťových front
\item \textbf{Radius} - sekcia obsahuje nastavenie autentizácie Radius 
\item \textbf{Routing} - sekcia obsahuje možnosti dynamického smerovania 
\item \textbf{Switch} - sekcia obsahuje konfiguráciu prepínača
\item \textbf{System} - sekcia obsahuje časť konfigurácie systémových nástrojov,
\item \textbf{Tools} - sekcia obshauje konfiguráciu mikrotik nástrojov
\item \textbf{Wireless} - sekcia obsahuje konfiguráciu bezdrátového rozhrania
\item \textbf{loginGui} - sekcia obsahuje súbory pre GUI, najdôležitejší je spúšťací súbor aplikácie \textit{loginGui.py}
\end{itemize}


%% Konec dokumentu
\end{document}